\documentclass[10pt]{article}
\usepackage[english]{babel} % Language to be used
\usepackage{natbib}
\usepackage{url}
\usepackage{hyperref}
\usepackage{listings}
\usepackage[usenames,dvipsnames]{color}    

\hypersetup{
  pdftitle={Conda \#1},
  pdfauthor={Wim R.M. Cardoen},
  pdfpagelayout=SinglePage,
  bookmarks=false,
  citecolor=black,
  colorlinks=true,
  linkcolor=black
}
\urlstyle{same}

\lstset{basicstyle=\ttfamily,
  tabsize=2,
  frame=single,
  showstringspaces=false,
  showspaces=false,
  showtabs=false,
  captionpos=b,
  breaklines=true,
  breakatwhitespace=false,        % sets if automatic breaks should only happen at whitespace
  keywordstyle=\color{RoyalBlue},      % keyword style
  commentstyle=\color{ForestGreen},   % comment style
  stringstyle=\color{ForestGreen}   
}

% Parameters to the margins 
\setlength{\topmargin}{-0.50in}
\setlength{\oddsidemargin}{-0.25in}
\setlength{\evensidemargin}{-0.25in}
\setlength{\textwidth}{7.0in}
\setlength{\textheight}{9.00in}


\begin{document}
\title{\centering{Conda \& Reproducibility}}
\author{Wim R.M. Cardoen \\ Email: \$(prefix)[at]gmail[dot]com \\ where \\ prefix='wcardoen' }
\date{\today}
\maketitle
\thispagestyle{empty}
% ----------------------------------------------------------------
\pagestyle{plain}
\pagenumbering{arabic} 
\setcounter{page}{1}
\renewcommand \thesection{\Roman{section}} 

\section{Reproducible science using conda}
Conda environments can be used to make python software installations reproducible, i.e. to generate 
python environments which bear identical version numbers for their python packages. 

In a first step, we will create an environment using the latest version of miniconda3\footnote{can be exchanged interchangeably for anaconda3 for every
instance in the document} 
installed on a machine running the Ubuntu 18.04 OS. Subsequently, the settings of this newly created conda environment will be exported into yaml file.
In the final step, we will load the settings in a new conda environment on a CHPC node running the Rocky8 OS.

\subsection{Creation of the reproducible environment} 

We assume that miniconda3 has been installed and has been set up in such a way that it allows to support 
more than one conda environment (the default conda environment bears the name \texttt{base}).

The most common way to support multiple conda environments is to invoke \lstinline[language=bash]{conda init} post the miniconda3 installation.
Unfortunately, this procedure modifies a user's existing startup shell (\lstinline[language=bash]{.bashrc/.tcshrc}) on a \textbf{permanent} basis: 
it appends a block of shell code in the startup shell and forces the latest miniconda3 installation to become the default.
At best, this approach may serve an individual user, but it is not apt for CHPC's cluster environment.

Therefore, we strongly recommend to use \href{https://github.com/CHPC-UofU/anaconda-modules/blob/master/miniconda3/latest.lua}{CHPC's lua module template}
which does not modify the startup shell, supports multiple conda environments and allows the user the flexibility to maintain different miniconda 
distributions next to one another.

In the following coding block, an environment with the name \texttt{genscience} is created. The command \texttt{conda activate genscience} allows
one to enter the \texttt{genscience} environment. Subsequent an array of Python packages (\texttt{numpy,\ldots, statsmodels}) as well as 
\texttt{texlive-core} will be installed in the \texttt{genscience} environment.
The command \lstinline[language=bash]{conda deactivate} forces one to leave the \texttt{genscience} environment and return to the \texttt{base}
environment.

\begin{lstlisting}[language=bash]
# Create a simple env: genscience
module load py39_4.12.0
# Create a simple env: genscience
conda create -y -n genscience 
# Activate the genscience environment
conda activate genscience
# Install an array of packages
conda install -y -c anaconda numpy scipy matplotlib 
conda install -y -c anaconda pandas scikit-learn scikit-learn-intelex
conda install -y -c anaconda jupyter
conda install -y -c anaconda xarray
conda install -y -c bokeh bokeh
conda install -y -c conda-forge dask
conda install -y -c conda-forge gdal
conda install -y -c conda-forge jupyterlab voila
conda install -y -c conda-forge jupyterlab-latex
conda install -y -c conda-forge scikit-image
conda install -y -c conda-forge statsmodels
conda install -y -c conda-forge texlive-core
# Deactivate the genscience environment
conda deactivate
\end{lstlisting}

The command \texttt{conda list} displays the environments that are accessible to the miniconda3 installation.
The symbol ('*') prepend the directory where the currently activated environment is stored.


\begin{lstlisting}[language=bash]
# Conda List:
# ----------
conda list

# Find all the Conda envs:
# -----------------------
base                  *  /home/sleipnir/software/pkg/mini3/py39_4.12.0
genscience               /home/sleipnir/software/pkg/mini3/py39_4.12.0/envs/genscience


sleipnir@ragnarok:~$ conda activate genscience
(genscience) sleipnir@ragnarok:~$ conda env list
# conda environments:
#
base                     /home/sleipnir/software/pkg/mini3/py39_4.12.0
genscience            *  /home/sleipnir/software/pkg/mini3/py39_4.12.0/envs/genscience
conda deactivate
\end{lstlisting}

\subsection{Export of a conda environment}
The information of the \texttt{genscience} environment can be easily stored in a file.
The following command stores the details of the \texttt{genscience} environment in the 
yaml file \texttt{genscience.py39.yml}.

\begin{lstlisting}[language=bash]
# Export the genscience env into a YAML file:
# -------------------------------------------
conda env export -v -n genscience -f genscience.py39.yml
\end{lstlisting}

\subsection{Reproduce an environement based on a YAML file}
In what follows we will recreate our environment on a new machine.
After loading an existing anaconda distribution we are able to generate a
Python environment based on the previously exported yaml file.

\begin{lstlisting}[language=bash]
module use ~/EigenModules
module load myanaconda/2020.11

[u0253283@kingspeak5:~]$ conda env list
# conda environments:
#
base                  *  /uufs/chpc.utah.edu/common/home/u0253283/software/pkg/anaconda3/2020.11

conda env create -n genscience -f $HOME/genscience.py39.yml

(genscience) [u0253283@kingspeak5:~]$ python3
Python 3.10.4 (main, Mar 31 2022, 08:41:55) [GCC 7.5.0] on linux
Type "help", "copyright", "credits" or "license" for more information.
>>> import numpy as np
>>> np.__version__
'1.22.3'
>>> import statsmodels
>>> statsmodels.__version__
'0.13.2'

\end{lstlisting}


% ----------------------------------------------------------------
\bibliographystyle{alpha}
\bibliography{ana}
\end{document}
% ----------------------------------------------------------------
