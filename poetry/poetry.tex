\documentclass[10pt]{article}
\usepackage[english]{babel} % Language to be used
\usepackage{natbib}
\usepackage{url}
\usepackage{hyperref}
\usepackage{listings}
\usepackage[usenames,dvipsnames]{color}    

\hypersetup{
  pdftitle={Poetry},
  pdfauthor={Wim R.M. Cardoen},
  pdfpagelayout=SinglePage,
  bookmarks=false,
  citecolor=black,
  colorlinks=true,
  linkcolor=black
}
\urlstyle{same}

\lstset{basicstyle=\ttfamily,
  tabsize=2,
  frame=single,
  showstringspaces=false,
  showspaces=false,
  showtabs=false,
  captionpos=b,
  breaklines=true,
  breakatwhitespace=false,        % sets if automatic breaks should only happen at whitespace
  keywordstyle=\color{RoyalBlue},      % keyword style
  commentstyle=\color{ForestGreen},   % comment style
  stringstyle=\color{ForestGreen}   
}

\newcommand{\POETRY}{\texttt{Poetry}}

% Parameters to the margins 
\setlength{\topmargin}{-0.50in}
\setlength{\oddsidemargin}{-0.25in}
\setlength{\evensidemargin}{-0.25in}
\setlength{\textwidth}{7.0in}
\setlength{\textheight}{9.00in}


\begin{document}
\title{\centering{Poetry}}
\author{Wim R.M. Cardoen \\ Email: \$(prefix)[at]gmail[dot]com \\ where \\ prefix='wcardoen' }
\date{\today}
\maketitle
\thispagestyle{empty}
% ----------------------------------------------------------------
\pagestyle{plain}
\pagenumbering{arabic} 
\setcounter{page}{1}
\renewcommand \thesection{\Roman{section}} 

In the following paragraphs we discuss the \POETRY\\,\cite{POETRY:2023} tool which handles
dependency management and packaging in Python.
The installation of \POETRY\ requires Python\,$>3.8$, but the tool is available for the \texttt{Linux},
\texttt{MacOS}, \texttt{Windows} OS.
All code below was executed on an laptop running on an \texttt{Ubuntu $22.04$} OS, Python $3.10.12$ 
and \texttt{Lmod} $8.7.31$.

\section*{Introduction}
Why poetry?

\section{Installation of }
Here comes the installation of Poetry

\section{Creation of a new Project}
Creation of a new project

\section{Add external dependencies}
Add external dependencies

\section{Build the package}
Build a package

\section{Publish the package}
Publish a package

% ----------------------------------------------------------------
\bibliographystyle{alpha}
\bibliography{poetry}
\end{document}
% ----------------------------------------------------------------
