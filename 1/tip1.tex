\documentclass[10pt]{article}
\usepackage[english]{babel} % Language to be used
\usepackage{natbib}
\usepackage{url}
\usepackage{hyperref}
\usepackage{listings}
\usepackage[usenames,dvipsnames]{color}    

\hypersetup{
  pdftitle={Python Tips \#1},
  pdfauthor={Wim R.M. Cardoen},
  pdfpagelayout=SinglePage,
  bookmarks=false,
  citecolor=black,
  colorlinks=true,
  linkcolor=black
}
\urlstyle{same}

\lstset{basicstyle=\ttfamily,
  tabsize=2,
  frame=single,
  showstringspaces=false,
  showspaces=false,
  showtabs=false,
  captionpos=b,
  breaklines=true,
  breakatwhitespace=false,        % sets if automatic breaks should only happen at whitespace
  keywordstyle=\color{RoyalBlue},      % keyword style
  commentstyle=\color{ForestGreen},   % comment style
  stringstyle=\color{ForestGreen}   
}

% Parameters to the margins 
\setlength{\topmargin}{-0.50in}
\setlength{\oddsidemargin}{-0.25in}
\setlength{\evensidemargin}{-0.25in}
\setlength{\textwidth}{7.0in}
\setlength{\textheight}{9.00in}


\begin{document}
\title{\centering{Python Tips \#1}}
\author{Wim R.M. Cardoen \\ Email: \$(prefix)[at]gmail[dot]com \\ where \\ prefix='wcardoen' }
\date{\today}
\maketitle
\thispagestyle{empty}
% ----------------------------------------------------------------
\pagestyle{plain}
\pagenumbering{arabic} 
\setcounter{page}{1}
\renewcommand \thesection{\Roman{section}} 

In the following paragraphs we discuss some Python tricks.
All code snippets below require Python\,$>=3.8$.  
They can be downloaded from the following location:\newline
\href{https://github.com/wcardoen/python-reflections.git}{https://github.com/wcardoen/python-reflections}.


\begin{itemize}
	\item F-strings:\newline		
The \lstinline[language=python]{format} method for the string class 
has been around since Python\,$2.7$. 
This is a familiar pattern for languages of which the syntax has been modelled on
the C language.
\begin{lstlisting}[language=python] 
# Example 1: format method since Python 2.7
artist={"von Beethoven":"Bonn",
        "de Balzac":"Tours",
        "van Eyck":"Maaseik",
        "Alighieri":"Firenze"}
for person in artist.keys():
    print("{0:>13} was born in {1}".format(person,artist[person]))
\end{lstlisting}

The aforementioned lines results in the following output:
\begin{verbatim}
von Beethoven was born in Bonn
    de Balzac was born in Tours
     van Eyck was born in Maaseik
    Alighieri was born in Firenze
\end{verbatim}

In Python\,$3.6$ the formatted string literal (\textbf{f-string}) was introduced.
In an f-string the following modifications have been applied to a regular string (and its \lstinline[language=python]{format} method):
\begin{enumerate}
	\item the \lstinline[language=python]{format} method applied to the regular string is omitted.
	\item the regular string must be prefixed by the '\textbf{f}' or '\textbf{F}' character.
        \item the indices referring to variables in the format method are replaced by the variables as such.	
\end{enumerate}
Therefore, the introduction of the f-string makes codes shorter and more readable.

\begin{lstlisting}[language=python]
# Example 1: use of f-strings (iter 1)
for person in artist.keys():
    print(f"{person:>13} was born in {artist[person]}")
\end{lstlisting} 

The format specifiers can also contain evaluated expressions:
\begin{lstlisting}[language=python]
# Example 1: use of f-strings (iter 2)
WIDTH = max(( len(item) for item in artist.keys()))
for person in artist.keys():
    print(f"{person:>{WIDTH}} was born in {artist[person]}")
\end{lstlisting}


Since \href{https://docs.python.org/3.8/whatsnew/3.8.html}{Python\,$3.8$} f-strings support the '\textbf{=}' character for self-documenting expressions and debugging.
The f-string \lstinline[language=python]|f"{expr=}"| will print the string 'expr=' and suffix it with the evaluated value of the expression 'expr'.

\begin{lstlisting}[language=python]
# Example 2: Self-documenting expression (iter 1)
from math import cos, pi
print(f" {cos(pi/4.0)=}")
\end{lstlisting}

This results into the following output:
\begin{verbatim}
 cos(pi/4.0)=0.7071067811865476
\end{verbatim}

Applying a format specifier to the previous example:
\begin{lstlisting}[language=python]
# Example 2: Self-documenting expression using a format specifier (iter 2)
WIDTH=10
PRECISION=4
from math import cos, pi
print(f" {cos(pi/4.0)=:{WIDTH}.{PRECISION}}")
\end{lstlisting}
We now get:
\begin{verbatim}
 cos(pi/4.0)=    0.7071
\end{verbatim}

For more info: \href{https://www.python.org/dev/peps/pep-0498/}{PEP-$0498$}.

\item Chaining of comparison operators:\newline

Let $X:=\{ x_1,\,x_2,\,x_3,\, \ldots,\,x_n \}$ be the set of Python expressions and $P:=\{ \widehat{op_1}, \,\widehat{op_2}, \, \ldots, \, \widehat{op_{n-1}} \}$
be the set of Python comparison operators \footnote{The Python language has the following comparison operators:\lstinline[language=python]{<,>,==,>=,<=,!=,is (not), (not) in}} applied to $X$.

The compound logical expression:
\begin{center}
$x_1 \; \widehat{op_1} \; x_2$ \lstinline[language=python]{and}
               $x_2 \; \widehat{op_2} \; x_3$ \lstinline[language=python]{and} $ \ldots$  \lstinline[language=python]{and}
$       x_{(n-1)} \; \widehat{op_{(n-1)}} \; x_n $ \newline
\end{center}
and
\begin{equation}
	x_1 \; \widehat{op_1} \; x_2  \; \widehat{op_2} \; x_3  \; \widehat{op_3} \; \ldots \; \widehat{op_{(n-1)}} \; x_n
\end{equation}	
are equivalent in the Python language.

Example $1$::
\begin{lstlisting}[language=python]
a,b,c=5,3,7
print(f"{a<b<c=}")
# Equivalent to: (a<b) and (b<c)
#                FALSE and TRUE => FALSE
\end{lstlisting}
which returns:
\begin{verbatim}
a<b<c=False
\end{verbatim}


Example $2$::
\begin{lstlisting}[language=python]
print(f"{15%4==3>2=}")
# Equivalent to: (15%4==3) and (3>2)
#            or:  TRUE     and TRUE => TRUE
\end{lstlisting}
which returns:
\begin{verbatim}
15%4==3>2=True
\end{verbatim}


Example $3$::
\begin{lstlisting}[language=python]
lstA=[[1,2],["hello","world"]]
lstB=lstA
lstB[1][0]="HELLO"
lstC=lstB[:]
print(f"{lstA is lstB is lstC=}")
print(f"{lstA == lstB == lstC=}")
# Equivalent to: (lstA is lstB) and (lstB is lstC)
#            or: TRUE and FALSE => FALSE
# Equivalent to: (lstA == lstB) and (lstB == lstC)
#            or: TRUE and TRUE => TRUE
\end{lstlisting}
which returns:
\begin{verbatim}
lstA is lstB is lstC=False
lstA == lstB == lstC=True
\end{verbatim}

\item The \lstinline[language=python]{itertools} module:\newline
The \lstinline[language=python]{enumerate} method generates the indices and values of an iterable's elements.
\begin{lstlisting}[language=python]
a= ['a','b','c']
b= ['x','y','z']
c= [10,100]
# enumerate: returns indices and el.
for index, item in enumerate(b):
    print(f"  El. with {index=} is {item=}")
\end{lstlisting}
\begin{verbatim}
  El. with index=0 is item='x'
  El. with index=1 is item='y'
  El. with index=2 is item='z'
\end{verbatim} 

Python's \lstinline[language=python]{zip} function generates an iterable of tuples
which is derived from its constituent iterables. The newly created iterable ends as soon as
one of its constituent iterables is exhausted.
\begin{lstlisting}[language=python]
for item in zip(a,b,c):
    print(f"  {item=}")
\end{lstlisting}
\begin{verbatim} 
  item=('a', 'x', 10)
  item=('b', 'y', 100)
\end{verbatim}

The \lstinline[language=python]{zip} method allows the extension of 
\lstinline[language=python]{enumerate} to several iterables:
\begin{lstlisting}[language=python]
for item in zip(range(len(b)),b,c):
    print(f"  {item=}")
\end{lstlisting}
\begin{verbatim}
  item=(0, 'x', 10)
  item=(1, 'y', 100)
\end{verbatim}  

Python's \lstinline[language=python]{itertools} module covers a wide range of functionality.
The following 2 examples use functions from this \href{https://docs.python.org/3/library/itertools.html#module-itertools}{module}.

The function \lstinline[language=python]{zip_longest} generates (like \lstinline[language=python]{zip}) 
an interable of tuples derived from its constituent
iterables. The new created iterable is exhausted when its \textbf{longest} constituent iterable is exhausted.
The "empty" elements are filled up with \lstinline[language=python]{None}.
\begin{lstlisting}[language=python]
import itertools as it
for item in it.zip_longest(a,b,c):
    print(f"  {item=}")
\end{lstlisting}
\begin{verbatim}
  item=('a', 'x', 10)
  item=('b', 'y', 100)
  item=('c', 'z', None)
\end{verbatim}

The \lstinline[language=python]{product} function generates a Cartesian product iterable based on its constituent iterables.
\begin{lstlisting}[language=python]
s = it.product(a,b,c)
for item in s:
    print(f"  {item=}")
\end{lstlisting}
\begin{verbatim}
  item=('a', 'x', 10)
  item=('a', 'x', 100)
  item=('a', 'y', 10)
  item=('a', 'y', 100)
  item=('a', 'z', 10)
  item=('a', 'z', 100)
  item=('b', 'x', 10)
  item=('b', 'x', 100)
  item=('b', 'y', 10)
  item=('b', 'y', 100)
  item=('b', 'z', 10)
  item=('b', 'z', 100)
  item=('c', 'x', 10)
  item=('c', 'x', 100)
  item=('c', 'y', 10)
  item=('c', 'y', 100)
  item=('c', 'z', 10)
  item=('c', 'z', 100)
\end{verbatim} 
Note that the result of the last block code can also be generated using the following tuple comprehension: 
\begin{lstlisting}[language=python]
s= ((x,y,z) for x in a for y in b for z in c)
for item in s:
    print(f"  {item=}")
\end{lstlisting}


\end{itemize}
% ----------------------------------------------------------------
\bibliographystyle{alpha}
\bibliography{ana}
\end{document}
% ----------------------------------------------------------------
