\documentclass[10pt]{article}
\usepackage[english]{babel} % Language to be used
\usepackage{natbib}
\usepackage{url}
\usepackage{hyperref}
\usepackage{listings}
\usepackage{amsfonts}
\usepackage[usenames,dvipsnames]{color}    

\hypersetup{
  pdftitle={Python Tips \#2},
  pdfauthor={Wim R.M. Cardoen},
  pdfpagelayout=SinglePage,
  bookmarks=false,
  citecolor=black,
  colorlinks=true,
  linkcolor=black
}
\urlstyle{same}

\lstset{basicstyle=\ttfamily,
  tabsize=2,
  frame=single,
  showstringspaces=false,
  showspaces=false,
  showtabs=false,
  captionpos=b,
  breaklines=true,
  breakatwhitespace=false,        % sets if automatic breaks should only happen at whitespace
  keywordstyle=\color{RoyalBlue},      % keyword style
  commentstyle=\color{ForestGreen},   % comment style
  stringstyle=\color{ForestGreen}   
}

% Parameters to the margins 
\setlength{\topmargin}{-0.50in}
\setlength{\oddsidemargin}{-0.25in}
\setlength{\evensidemargin}{-0.25in}
\setlength{\textwidth}{7.0in}
\setlength{\textheight}{9.00in}


\begin{document}
\title{\centering{Python Tips \#2}}
\author{Wim R.M. Cardoen \\ Email: \$(prefix)[at]gmail[dot]com \\ where \\ prefix='wcardoen' }
\date{\today}
\maketitle
\thispagestyle{empty}
% ----------------------------------------------------------------
\pagestyle{plain}
\pagenumbering{arabic} 
\setcounter{page}{1}
\renewcommand \thesection{\Roman{section}} 

In the following paragraphs we discuss some recent innovations in Python. 
All code snippets were tested using Python\,$3.11.0$b$3$.
The examples can be downloaded from the following location:\newline
\href{https://github.com/wcardoen/python-reflections.git}{https://github.com/wcardoen/python-reflections}.


\begin{itemize}
	\item Match-operator:\newline
A lot of programming languages have selection control mechanisms 
		beyond the \lstinline[language=python]{if}, \lstinline[language=python]{elif}, \lstinline[language=python]{else} constructs. 
		Its C counterpart is the \lstinline[language=c]{switch} construct.

\begin{lstlisting}[language=python]
# Traditional if/elif/else statements
def find_capital(country):
    if country == 'France':
        return 'Paris'
    elif country == 'Germany':
        return 'Berlin'
    elif country == 'Netherlands':
        return 'Amsterdam'
    elif country == 'Belgium':
        return 'Brussels'
    else:
        return 'SORRY!'

for country in ['Belgium', 'Poland']:
    print(f"  Country:{country:>15s} -> Capital:{find_capital(country):>10s}")

\end{lstlisting}

The aforementioned code block results in the following output:
\begin{verbatim}
  Country:        Belgium -> Capital:  Brussels
  Country:         Poland -> Capital:    SORRY!
\end{verbatim}

In Python\,$3.10$, the \lstinline[language=python]{match} construct 
was introduced.
\begin{lstlisting}[language=python] 
def find_capital2(country):
# match/case construct (Python >= 3.10)
    match country:
        case 'France':
             return 'Paris'
        case 'Germany':
             return 'Berlin'
        case 'Netherlands':
             return 'Amsterdam'
        case 'Belgium':
             return 'Brussels'
        case _:
             return 'Sorry!'
    
for country in ['Belgium', 'Denmark']:
    print(f"  Country:{country:>15s} -> Capital:{find_capital2(country):>10s}")
\end{lstlisting}

The aforementioned block results into:
\begin{verbatim}
  Country:        Belgium -> Capital:  Brussels
  Country:        Denmark -> Capital:    Sorry!
\end{verbatim}

		You can combine several patterns using the  \lstinline[language=python]{|} (i.e.\,$\cup$ operator).
\begin{lstlisting}[language=python]
def find_continent(country):
    match country:
        case 'Belgium'|'France'|'Germany'|'Netherlands':
            return 'Europe'
        case 'China'|'India'|'Japan':
            return 'Asia'
        case _:
            return 'Sorry!'
for country in ['France','China','USA','India']:
    print(f"  Country:{country:>15s} -> Continent:{find_continent(country):>10s}")
\end{lstlisting}

The above code block produces the following output:
\begin{verbatim}
  Country:         France -> Continent:    Europe
  Country:          China -> Continent:      Asia
  Country:            USA -> Continent:    Sorry!
  Country:          India -> Continent:      Asia
\end{verbatim}  

Patterns can also be verified by unpacking:
\begin{lstlisting}[language=python]
def find_location(point):
    match point:
        case (0,0,0):
            return "Origin"
        case (x,0,0):
            return "Pt. on x-axis"
        case (0,y,0):
            return "Pt. on y-axis"
        case (0,0,z):
            return "Pt. on z-axis"
        case (x,y,0):
            return "Pt. in the xy-plane"
        case (x,0,z):
            return "Pt. in the xz-plane"
        case (0,y,z):
            return "Pt. in the yz-plane"
        case (x,y,z):
            return "Reg. pt."
        case _:
            return "NOT a 3D-point"
for item in [(3,4,5), [2,0,0], (0,0,0), (0,3,2), (3,4,5,6)]:
    print(f"  Pt.:{str(item):>15s}   Type:{find_location(item)}")
\end{lstlisting}

This results into:
\begin{verbatim}
  Pt.:      (3, 4, 5)   Type:Reg. pt.
  Pt.:      [2, 0, 0]   Type:Pt. on x-axis
  Pt.:      (0, 0, 0)   Type:Origin
  Pt.:      (0, 3, 2)   Type:Pt. in the yz-plane
  Pt.:   (3, 4, 5, 6)   Type:NOT a 3D-point
\end{verbatim}

The match pattern construct is wide topic. Three PEPS (\href{https://peps.python.org/pep-0622}{PEP-$622$}, 
\href{https://peps.python.org/pep-0634}{PEP-$634$}, \href{https://peps.python.org/pep-0636/}{PEP-$636$}) were written to address it.


\item Merging of dictionaries\newline
The merging and update of Python dictionaries has been improved in Python\,$3.9$ 
by introducing the \lstinline[language=python]{|} and \lstinline[language=python]{|=} operators. The details are discussed in  
\href{https://peps.python.org/pep-0584/}{\texttt{PEP-}$0584$}
\begin{lstlisting}[language=python] 
capitals1 = {'france':'paris', 'germany':'berlin'}
capitals2 = {'france':'paris', 'belgium':'brussels'}

# Merging: Creation of a new dict object
capitals3 = capitals1 | capitals2
print(f"  capitals3:\n{capitals3}")

# Update in-place operation
capitals1 |= capitals2
print(f"  capitals1:\n{capitals1}")
\end{lstlisting}

\begin{verbatim}
  capitals3:
{'france': 'paris', 'germany': 'berlin', 'belgium': 'brussels'}
  capitals1:
{'france': 'paris', 'germany': 'berlin', 'belgium': 'brussels'}
\end{verbatim}

\item Removing the prefixes/suffixes of strings\newline
Python\,$3.9$ introduced some handy methods to remove suffixes and prefixes 
(\href{https://peps.python.org/pep-0616/}{\texttt{PEP-}$0616$}).
\begin{lstlisting}[language=python]
city="Witwatersrand"
STR1, STR2 ="Wit", "rand"
print(f"String:'{city}'")
print(f"  remove the prefix '{STR1}'  ->  '{city.removeprefix(STR1)}'")
print(f"  remove the sufffix '{STR2}'  ->  '{city.removesuffix(STR2)}'")
\end{lstlisting}

\begin{verbatim}
String:'Witwatersrand'
  remove the prefix 'Wit'  ->  'watersrand'
  remove the sufffix 'rand'  ->  'Witwaters'
\end{verbatim}

\item \lstinline[language=python]{math} module \newline
The \lstinline[language=python]{math} module was extended with
some interesting methods, among them:
\begin{itemize}
\item \lstinline[language=python]{math.isqrt} : returns the integer part of the square root
\item \lstinline[language=python]{math.gcd}   : returns the Greatest Common Divisor
\item \lstinline[language=python]{math.lcm}   : returns the Least Common Multiple
\item \lstinline[language=python]{math.prod}  : calculates the products of the elements in a iterable
\item \lstinline[language=python]{math.comb}  : calculates the number of combinations $n \choose k$
\item \lstinline[language=python]{math.perm}  : calculates the number of permutations $\frac{n!}{(n-k)!}$
\item \lstinline[language=python]{math.dist}  : calculates the euclidean distance between $p,q \,\in \mathbb{R}^n$
\end{itemize}		

\begin{lstlisting}[language=python]
import math
print(f"  math.isqrt(26)    :{math.isqrt(26)}")
print(f"  math.gcd(24,12,36):{math.gcd(24,12,36)}")
print(f"  math.lcm(2,4,6,8) :{math.lcm(2,4,6,8)}")
print(f"  math.prod([2,3,4,5,6],start=10):{math.prod([2,3,4,5,6],start=10)}")
print(f"  math.comb(5,2):{math.comb(5,2)}")
print(f"  math.perm(5,2):{math.perm(5,2)}")
p = range(1,10)
q = range(2,11)
print(f"  math.dist(p,q):{math.dist(p,q)}")
\end{lstlisting}

which results into:
\begin{verbatim}
  math.isqrt(26)    :5
  math.gcd(24,12,36):12
  math.lcm(2,4,6,8) :24
  math.prod([2,3,4,5,6],start=10):7200
  math.comb(5,2):10
  math.perm(5,2):20
  math.dist(p,q):3.0
\end{verbatim}

\end{itemize}
% ----------------------------------------------------------------
\bibliographystyle{alpha}
\bibliography{ana}
\end{document}
% ----------------------------------------------------------------
